\documentclass[11pt,a4paper]{article}
\usepackage[UTF8]{ctex}
\usepackage{amsmath}
\title{流水线型FFT设计}
\author{刘润}

\begin{document}

\maketitle
\section{FFT混合基分解}

\subsection{FFT混合基分解一般形式}
对于$N$点$DFT$,如果$N$是一个复合数,它可以分解成一些因子的乘积,则可以用$FTT$的一般算法,即混合基$FFT$算法,
而基-2算法只是这种一般算法的特例。

若$N$可以表示为复合数$N=r_1r_2 \cdots r_L$,则对于$n<r_1r_2 \cdots r_L$,的任何一个正整数$n$,可以按照
$L$基$r_1$,$r_2$,$\cdots$,$r_L$ 表示为多基多进制形式$(n_{L-1}n_{L-2} \cdots n_1n_0)_{r_1r_2 \cdots r_L}$,
这一多基多进制所代表的数值为:
\begin{equation}
    (n)_{10} = n_{L-1}(r_2r_3 \cdots r_L) + n_{L-2}(r_3r_4 \cdots r_L) + \cdots + n_1n_L + n_0
\end{equation}
其倒位序形式为$[\rho(n)]_{r_Lr_{L-1} \cdots r_2r_1} = (n_0n_1 \cdots n_{L-2}n_{L-1})_{r_Lr_{L-1} \cdots r_2r_1}$,它所代表的数值为:
\begin{equation}
    [\rho(n)]_{10}=n_0(r_1r_2 \cdots r_{L-1}) + n_1(r_1r_2 \cdots r_{L-2}) + \cdots + n_{L-2}r_1 + n_{L-1}
\end{equation}
在这一多基多进制的表示中
\begin{equation}
\begin{aligned}
    &n_0 = 0,1,\cdots ,r_{L}-1 \\
    &n_1 = 0,1,\cdots ,r_{L-1}-1 \\
    &\cdots \\
    &n_{L-1} = 0,1,\cdots ,r_1-1 
\end{aligned}
\end{equation}


\subsection{512点FFT示例}
\end{document} 